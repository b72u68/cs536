\documentclass{article}
\usepackage[left=0.3in,right=0.3in,top=1in,bottom=1in]{geometry}
\usepackage{amsmath, amsthm, amssymb, verbatim, enumerate}
\usepackage{listings}
\usepackage{fancyhdr}
\usepackage{xcolor}
\usepackage{forest}

\newcommand{\z}{\mathbb{Z}}
\newcommand{\stmt}[1]{\text{\sffamily #1}}
\newcommand{\ifs}[3]{\stmt{if}\;#1\;\stmt{then}\;\{#2\}\;\stmt{else}\;\{#3\}}
\newcommand{\while}[1]{\stmt{while}\;(#1)\;\{}
\newcommand{\assign}[2]{#1 := #2}
\newcommand{\const}[1]{\overline{#1}}
\newcommand{\await}[2]{\stmt{await}\;(\ensuremath{#1})\;\stmt{then}\;\{#2\}}
\newcommand{\inv}[1]{\{\textbf{inv}\;\ensuremath{#1}\}}
\newcommand{\dec}[1]{\{\textbf{dec}\;\ensuremath{#1}\}}
\newcommand{\havoc}[1]{\ensuremath{\stmt{havoc}\; #1}}
\newcommand{\ands}{\stmt{\&\&}}
\newcommand{\cond}[3]{\ensuremath{#1\,?\,#2\,:\,#3}}
\newcommand{\skips}{\stmt{skip}}
\renewcommand{\labelenumi}{\alph{enumi})}
\newcommand{\wlp}[2]{\ensuremath{wlp(#1, #2)}}
\newcommand{\hoare}[3]{\{\,\ensuremath{#1}\,\}\;\ensuremath{#2}\;\{\,\ensuremath{#3}\,\}}
\newcommand{\wrapup}[2]{I spent about #1 hours on this homework, in total #2 hours of actual working time.}

\newtheoremstyle{task}%
    {}{}%
    {\normalfont}{}%
    {\bfseries}{\normalfont{.}}%
    { }%
    {\thmname{#1}\thmnumber{ #2}\thmnote{ (#3)}}

\theoremstyle{task}
\newtheorem{task}{Task}[section]

\pagestyle{fancy}
\fancyhf{}
\rhead{My Dinh}
\lhead{CS536: Science of Programming}
\cfoot{\thepage}

\title{IIT CS536: Science of Programming\\
  {\large Homework 7: Parallelism}}
\author{My Dinh}
\date{\today}

\begin{document}
\maketitle

\section{A simple parallel program}

\begin{task}[Written, 4 points]
\end{task}
\begin{center}
    \begin{tabular}{c | c | c | c | c}
        i & j & Vars(j) & Change(i) & i interferes w/ j? \\
        \hline
        $s_1$ & $s_2$ & $i2, n, r2$ & $i1, r1$ & No \\
        $s_2$ & $s_1$ & $i1, r1, n$ & $i2, r2$ & No \\
    \end{tabular}
\end{center}
From the table above, we know that $s_1$ and $s_2$ are disjoint. Thus, $s$
is a disjoint parallel program.

\begin{task}[Written, 7 points]
\end{task}
\[
    \begin{array}{l l}
        & \{n \geq 0 \} \\
        \assign{i1}{\const{0}}; & \\
        \assign{r1}{\const{0}}; & \{n \geq 0 \wedge i1 = 0 \wedge r1 = 0\} \\
        \inv{i1 \leq n/2 \wedge r1 = sum(0, i1)} & \\
        \while{i1 < n/\const{2}} & \\
        \quad \assign{r1}{r1 + i1}; & \\
        \quad \assign{i1}{i1 + \const{1}} \\
        \}; & \{r1 = sum(0, n/2)\} \\
        \assign{i2}{n/\const{2}}; & \\
        \assign{r2}{\const{0}}; & \{r1 = sum(0, n/2) \wedge i2 = n/2 \wedge r2 = 0\} \\
        \inv{i2 \leq n \wedge r2 = sum(n/2, i2) \wedge r1 = sum(0, n/2)} & \\
        \while{i2 < n} & \\
        \quad \assign{r2}{r2 + i2}; & \\
        \quad \assign{i2}{i2 + \const{1}}; & \\
        \}; & \{r1 = sum(0, n/2) \wedge r2 = sum(n/2, n)\} \\
            & \Rightarrow \{r1 + r2 = sum(0, n)\} \\
        \assign{r}{r1 + r2} & \\
        & \{r = sum(0, n)\}
    \end{array}
\]

\begin{task}[Written, 11 points]
\end{task}
Before we can use \textbf{Par rule} to prove this program, we have to prove
that thread $s_1$ and $s_2$ in this program have disjoint conditions.

\begin{center}
    \begin{tabular}{c | c | c | c | c | c | c}
        i & j & Change(i) & Vars(j) & FV(j) & i interferes w/ j? & i interferes w/ cond j? \\
        \hline
        $s_1$ & $s_2$ & $r1, i1$ & $r2, i2, n$ & $r2, i2, n$ & No & No \\
        $s_2$ & $s_1$ & $r2, i2$ & $i1, r1, n$ & $i1, r1, n$ & No & No \\
    \end{tabular}
\end{center}
Thus, $s_1$ and $s_2$ have disjoint conditions.

\[
    \begin{array}{l l}
        & \{n \geq 0 \} \\
        \lbrack & \{p_1 \equiv n \geq 0\} \\
        \assign{i1}{\const{0}}; & \\
        \assign{r1}{\const{0}}; & \{n \geq 0 \wedge i1 = 0 \wedge r1 = 0\} \\
        \inv{i1 \leq n/2 \wedge r1 = sum(0, i1)} & \\
        \while{i1 < n/\const{2}} & \\
        \quad \assign{r1}{r1 + i1}; & \\
        \quad \assign{i1}{i1 + \const{1}} \\
        \} & \{q_1 \equiv r1 = sum(0, n/2)\} \\
        || & \{p_2 \equiv n \geq 0\} \\
        \assign{i2}{n/\const{2}}; & \\
        \assign{r2}{\const{0}}; & \{n \geq 0 \wedge i2 = n/2 \wedge r2 = 0\} \\
        \inv{i2 \leq n \wedge r2 = sum(n/2, i2)} & \\
        \while{i2 < n} & \\
        \quad \assign{r2}{r2 + i2}; & \\
        \quad \assign{i2}{i2 + \const{1}}; & \\
        \} & \{q_2 \equiv r2 = sum(n/2, n)\} \\
        \rbrack; & \{r_1 = sum(0, n/2) \wedge r_2 = sum(n/2, n)\} \Rightarrow \{r1 + r2 = sum(0, n)\} \\
        \assign{r}{r1 + r2} & \\
        & \{r = sum(0, n)\}
    \end{array}
\]

\section{A More Realistic Parallel Program}

\begin{task}[Written, 10 points]
\end{task}
Proof outline for $s_1$:
\[
    \begin{array}{l l}
        & \{n \geq 0 \wedge i1 = 0 \wedge r1 = 0\} \\
        \inv{i1 \leq n/2 \wedge r1 = sum(0, i1)} & \\
        \while{i1 < n/\const{2}} & \{i1 < n/2 \wedge i1 \leq n/2 \wedge r1 = sum(0, i1)\} \Rightarrow \{i1 < n/2 \wedge r1 = sum(0, i1)\}\\
        \quad < \assign{r1}{r1 + i1}; & \{i1 < n/2 \wedge r1 = sum(0, i1) + i1\} \Rightarrow \{i1+1 < n/2+1 \wedge r1 = sum(0, i1+1)\}\\
        \quad \assign{i1}{i1 + \const{1}} > & \{i1 < n/2+1 \wedge r1 = sum(0, i1)\} \Rightarrow \{i1 \leq n/2 \wedge r1 = sum(0, i1)\}\\
        \} & \{i1 \geq n/2 \wedge i \leq n/2 \wedge r1 = sum(0, i1)\} \\
           & \Rightarrow \{r1 = sum(0, n/2)\}
    \end{array}
\]

\vspace{1em}
Proof outline for $s_2$:
\[
    \begin{array}{l l}
        & \{0 \leq i1 \leq n/2 \wedge i2 = n/2 \wedge r1 = sum(0, i1) \wedge r2 = 0\} \\
        \inv{i1 \leq n/2 \wedge i2 \leq n \wedge r1 = sum(0, i1) \wedge \\
        \quad r2 = sum(n/2, i2)} & \\
        \while{i2 < n} & \{i2 < n \wedge i1 \leq n/2 \wedge i2 \leq n \wedge r1 = sum(0, i1) \wedge r2 = sum(n/2, i2)\} \\
                       & \Rightarrow \{i2 < n \wedge i1 \leq n/2 \wedge r1 = sum(0, i1) \wedge r2 = sum(n/2, i2)\} \\
        \quad \assign{r2}{r2 + i2}; & \{i2 + 1 < n + 1 \wedge i1 \leq n/2 \wedge r1 = sum(0, i1) \wedge r2 = sum(n/2, i2+1)\} \\
        \quad \assign{i2}{i2 + \const{1}} & \{i2 < n + 1 \wedge i1 \leq n/2 \wedge r1 = sum(0, i1) \wedge r2 = sum(n/2, i2)\} \\
                                          & \Rightarrow \{i2 \leq n \wedge i1 \leq n/2 \wedge r1 = sum(0, i1) \wedge r2 = sum(n/2, i2)\} \\
        \}; & \{i2 \geq n \wedge i2 \leq n \wedge i1 \leq n/2 \wedge r1 = sum(0, i1) \wedge r2 = sum(n/2, i2)\} \\
            & \Rightarrow \{i1 \leq n/2 \wedge r1 = sum(0, i1) \wedge r2 = sum(n/2, n)\} \\
        \inv{i1 \leq n/2 \wedge r1 = sum(0, i1) \wedge r2 = sum(n/2, n)} & \\
        \while{i1 < n/\const{2}} \skips\}; & \{r1 = sum(0, n/2) \wedge r2 = sum(n/2, n)\} \\
                                           & \Rightarrow \{r1 + r2 = sum(0, n)\} \\
        \assign{r}{r1 + r2} & \{r1 + r2 = sum(0, n) \wedge r = r1 + r2\} \\
                            & \Rightarrow \{r = sum(0, n)\}
    \end{array}
\]

\pagebreak
\begin{task}[Written, 6 points]
\end{task}
Because $s_1$ and $s_2$ are disjoint programs and have disjoint conditions
so the proof $s_1*$ and $s_2*$ does not interfere with each other. We can
see that proof $s_1*$ reads $i1, r1$, $n$  and writes $i1, r1$ that do not
change the conditions and proof invariants in proof $s_2*$ (because the changes
are kept in a critical section). And proof $s_2*$ reads $r1, i1, r2, i2, n$ and
writes $s2, i2, r$ that do not interfere proof $s_1*$.

\section{Even More Realistic with Await}

\begin{task}[Written, 3 points]
\end{task}
\[
    \begin{array}{l l}
        \lbrack & \\
        \while{i1 < n/\const{2}} & \\
        \quad \langle \assign{r1}{r1 + i1}; & \\
        \quad \assign{i1}{i1 + \const{1}} \rangle & \\
        \} & \\
        || & \\
        \while{i2 < n} & \\
        \quad \assign{r2}{r2 + i2}; & \\
        \quad \assign{i2}{i2 + \const{1}} & \\
        \}; & \\
        \await{i1 \geq n/\const{2}}{\assign{r}{r1 + r2}} & \\
        \rbrack &
    \end{array}
\]

\begin{task}[Written, 2 points]
\end{task}

Because a $\stmt{while}$ loop might keep running in one thread that prevents
other thread from running, which costs a lot of resources. It can also cause the
program to diverge and make the total correctness of no longer hold. While
$\stmt{await}$ check the condition, if it false then it will let other threads
run. It only runs the statement when the condition is true.

\section{A Buggy (and therefore even more realistic) Parallel Program}

\begin{task}[Written, 7 points]
\end{task}

\begin{enumerate}
    \item With the given precondition and initial state, program $s$ can execute
        thread 2 until the end without having to wait for thread 1 to finish
        to compute the final sum $r$. When it reaches $\stmt{while } i1 < n/\const{2}\;
        \{\skips\}$ in thread 2, because thread 1 hasn't run yet, the value of
        $i1 = 8285$ (from initial state $\sigma$) is larger than $n/2$ so the
        program in thread 2 exits the loops. At this point, if thread 2 computes
        $r = r1 + r2$ before thread 1 can end, the value of $r$ is either 524999
        ($r = 523752 + sum(13, 26)$ for $r1 = 523752$ from state $\sigma$) or
        from $sum(0, 0) + sum(13, 26)$ to $sum(0, 12) + sum(13, 26)$ (thread
        2 interleaves to thread 1 then back to thread 2).

    \item In hint 1 given in part a), there are 78 execution paths that will have
        this behavior (incorrect paths). Thus, the expected number of times
        we would have to run the program to find the bug in testing is
        \begin{equation*}
            \frac{\text{\# of possible execution paths}}{\text{\# of incorrect execution paths}}
            = \frac{1.3 \times 10^{14}}{78} = 1.67 \times 10^{12} \text{ times}
        \end{equation*}
        The total time it would take to find one of the buggy execution described
        in part a) is
        \begin{equation*}
            \frac{1.67 \times 10^{12}}{1000} = 1.67 \times 10^9 \text{ seconds} \approx 52.85 \text{ years}
        \end{equation*}
    \item The total time an user will encounter the bug, in expectation, is
        \begin{equation*}
            \frac{1.67 \times 10^{12}}{10^{11}} = 16.7 \text{ days}
        \end{equation*}
    \item Because program verification helps us make sure that our program will
        work correctly as we've intended it to do and avoid possible errors in
        the program. In the above example, without program verification, even
        though the error/bug does not affect us right away, it might appear in
        the future and cause the users and developers some troubles.
\end{enumerate}

\section{One more wrap-up question}

\begin{task}[Written, 0 points]
\end{task}
\wrapup{3}{2}


\end{document}
